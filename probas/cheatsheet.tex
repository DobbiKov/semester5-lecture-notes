\documentclass[a4paper]{article}

\usepackage[utf8]{inputenc}
\usepackage[T1]{fontenc}
\usepackage{textcomp}
\usepackage[english]{babel}
\usepackage{amsmath, amssymb, amsthm}
\usepackage{float}


% figure support
\usepackage{import}
\usepackage{xifthen}
\pdfminorversion=7
\usepackage{pdfpages}
\usepackage{transparent}
\usepackage{hyperref}
\usepackage[margin=0.8in]{geometry}

\usepackage{setspace}
\setlength{\parindent}{0in}

\newcommand{\incfig}[1]{%
    \def\svgwidth{\columnwidth}
    \import{./figures/}{#1.pdf_tex}
}

\pdfsuppresswarningpagegroup=1

\newcommand{\N}{\mathbb{N}}
\newcommand{\R}{\mathbb{R}}
\newcommand{\Z}{\mathbb{Z}}
\newcommand{\Q}{\mathbb{Q}}

\newtheorem{theorem}{Théorème}[section]
\newtheorem{definition}{Définition}[section]
\newtheorem{eg}{Exemple}[section]
\newtheorem{prop}{Proposition}[section]
\newtheorem{propriete}{Propriété(s)}[section]
\newtheorem*{notation}{Notation}
\newtheorem*{remarque}{Remarque}

\title{CheatSheet de Probabilité Continue}
\author{Yehor Korotenko}
\date{\today}

\begin{document}
\maketitle
\section{Mathématiques Générales} 
\subsection{Binomial}
\[
    \binom{n}{k} = \frac{n!}{k!(n-k)!}
\] 
Propriétés utiles:
\begin{enumerate}
    \item $\binom{n}{0} = \binom{n}{n} = 1$
    \item $\binom{n}{1} = n$ 
    \item $\binom{n}{k} = \binom{n-1}{k-1} + \binom{n-1}{k}$
\end{enumerate}

\section{Les loi à densité}%
\label{sec:Les loi à densité}
\subsection{Normale / Gaussienne}
La sommes du grand nombre des v.a.s tend vers la v.a qui suit la loi normale.

Soit $Z \sim \mathcal{N}(0, 1)$, alors
 \[
     f_Z(z) = \phi(z) = \frac{1}{\sqrt{2 \pi} }e^{-\frac{z^2}{2}}
\] 
Cette loi s'appelle loi normale standarte.

Soit $X = \mu Z + \sigma$, alors  $X \sim \mathcal{N}(\mu, \sigma^2)$ et 
 \[
     f_X(z) = \frac{1}{\sigma \sqrt{2 \pi} }e^{- \frac{(x - \mu)^2}{2\sigma^2}}
\] 

L'esperance et la variance sont:
\begin{align*}
    E[X] = \mu & & Var(X) = \sigma^2
\end{align*}

\textbf{Propriétés utiles}.
Les propriétés suivantes s'appliques à la loi normale standarte.
\begin{enumerate}
    \item Si $Z \sim \mathcal{N}(0, 1)$, alors  $-Z \sim \mathcal{N}(0, 1)$ 
    \item Si $Z \sim \mathcal{N}(0, 1)$ et  $\Phi(z)$ est son CDF, alors
         \[
        \Phi(z) = 1 - \Phi(-z)
        \] 
\end{enumerate}

\subsection{Loi exponentielle}
Cette loi est un équivalent continue de la loi géomètrique discrète.
La v.a qui suit cette loi dit combien de temps il reste d'attendre avant que le
premier succès arrive.

Soit $X \sim Exp(\lambda)$, alors
 \[
     f_X(z) = \lambda e^{-\lambda z}
\] 
L'esperance et la variance sont:
\begin{align*}
    E[X] = \frac{1}{\lambda} & & Var(X) = \frac{1}{\lambda^2}
\end{align*}
\textbf{Propriétés utiles}.
\begin{enumerate}
    \item Si $X \sim Exp(\lambda)$, alors
         \[
        P(X \ge s + t | X \ge s) = P(X \ge t)
        \] 
        \textbf{Intuition}: Une ampoule de durée de vie exponentielle :
        même si elle a déjà duré 3 heures, la probabilité qu’elle tienne
        encore 2 heures est la même que pour une ampoule toute neuve.
\end{enumerate}

\subsection{Loi Beta}
\texttt{Probabilité des probabilité}. On utilise cette loi quand on veut
estimer ou calculer le paramètre d'une probabilité en fonctions de nombres de
succès et échecs.

Si $X \sim Beta(\alpha, \beta)$ alors:
\[
    f_X(z) = \frac{1}{B(\alpha, \beta)}x^{\alpha-1}(1 - x)^{\beta - 1}
\] 
où
\[
    B(\alpha, \beta) = \int_{{0}}^{{1}} {t^{\alpha-1} (1-t)^{\beta - 1}} \: d{t} {}
\] 

L'esperance et variance sont:
 \begin{align*}
     E[X] = \frac{\alpha}{\alpha + \beta} & & Var(X) = \frac{\alpha \beta}{(\alpha + \beta)^2 (\alpha + \beta + 1)}
\end{align*}

\subsection{Loi Gamma}
Si la loi exponentielle dit combien de temps il faut attendre avant que le
succès arrive, alors la loi Gamma dit combien de temps il faut attendre avant
que le $k^{\text{ième}}$ succès arrive.

Alors si $X \sim Gamma(\alpha, \beta)$ (notons aussi $r = \alpha$ et  $\lambda = \beta$), alors:
\[
    f(t) = \frac{\lambda^{\alpha}}{\Gamma(\alpha)}t^{\alpha-1}e^{-\beta t}
\] 

dans le cas où $\alpha = r \in \N$:
\[
    f(t) = e^{-\lambda t} \frac{\lambda^r t^{r-1}}{(r-1)!}
\] 

L'esperance et variance sont:
\begin{align*}
    E[X] = \frac{\alpha}{\beta} & & Var(X) = \frac{\alpha}{\beta^2}
\end{align*}

\end{document}
