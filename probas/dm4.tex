\documentclass[a4paper,12pt]{article}
\usepackage[utf8]{inputenc}
\usepackage[T1]{fontenc}
\usepackage[french]{babel}
\usepackage{amsmath, amssymb}

\begin{document}

Une variable aléatoire réelle \( X \) est dite distribuée sur le réseau \( \alpha \mathbb{Z} \), où \( \alpha > 0 \), si
\[
P(X \in \alpha \mathbb{Z}) = 1 \quad \text{et} \quad \forall \beta > \alpha, \; P(X \in \beta \mathbb{Z}) < 1.
\]

De manière équivalente, \( \alpha \mathbb{Z} \) est le plus petit réseau tel que \( P(X \in \alpha \mathbb{Z}) = 1 \). 
S’il n’existe pas de tel nombre réel \( \alpha \), la v.a. \( X \) est dite non-réseau, et nous convenons de dire qu’elle est distribuée sur \( 0\mathbb{Z} \), que nous identifions à \( \mathbb{R} \).

\bigskip
0) Cette convention vous paraît-elle cohérente?

\medskip

Dans tout le problème, nous considérons une suite \( (X_n)_{n \in \mathbb{N}^*} \) de v.a. réelles i.i.d. distribuée sur le réseau \( \alpha \mathbb{Z} \), où \( \alpha \geq 0 \).  
Nous construisons la marche aléatoire \( (S_n)_{n \in \mathbb{N}} \) associée, définie par :

\[
\forall n \in \mathbb{N}, \quad S_n = X_1 + \cdots + X_n.
\]

\medskip

Un point \( x \in \mathbb{R} \) est dit \textbf{récurent} si :

\[
\forall \varepsilon > 0, \quad P(|S_n - x| < \varepsilon \text{ infiniment souvent}) = 1.
\]

\medskip

Un point \( x \in \mathbb{R} \) est dit \textbf{possible} si :

\[
\forall \varepsilon > 0, \ \exists k \in \mathbb{N}, \quad P(|S_k - x| < \varepsilon) > 0.
\]

\bigskip

\noindent \textbf{Partie I : Généralités.}

\medskip

\noindent 1) a) Montrez que l’ensemble des points récurrents est une partie fermée de \( \mathbb{R} \).

\smallskip

\noindent b) Montrez qu’un point possible appartient à \( \alpha \mathbb{Z} \).
\textbf{2) a)} Montrez que, pour tous $\varepsilon > 0$, $n, k \geq 1$,

\[
\{\, |S_k - y| < \varepsilon \,\} \cap 
\{\, |S_{n+k} - S_k - (x - y)| < 2\varepsilon \ \text{f.s.} \,\}
\subset 
\{\, |S_n - x| < \varepsilon \ \text{f.s..} \,\},
\]

où \textit{f.s.} signifie \textit{finitement souvent}.

3) \quad
a) \; Rappeler à quelles conditions une partie de $\mathbb{R}$ est un sous-groupe de $\mathbb{R}$ pour l’addition.

b) \; Montrez que l’ensemble des points récurrents est \sout{un sous-groupe fermé de $\mathbb{R}$} soit vide soit un groupe.

Nous admettrons le résultat suivant : les sous-groupes fermés de $\mathbb{R}$ sont exactement les réseaux
$\beta \mathbb{Z}$, $\beta \ge 0$.

c) \; Montrez que $\alpha \mathbb{Z}$ est le plus petit sous-groupe fermé de $\mathbb{R}$ qui contient tous les états possibles.

4) \quad Supposons que la marche aléatoire $(S_n)_{n \in \mathbb{N}}$ admette un point récurrent. Montrez que :

\textbf{a)} $0$ est récurrent;\\
\textbf{b)} si $y$ est possible, alors $-y$ est récurrent;\\
\textbf{c)} tous les points de $\alpha \mathbb{Z}$ sont récurrents.\\[1em]

\section*{Partie II : Un critère de récurrence}

Soit $I$ un intervalle borné de $\mathbb{R}$ tel que $I \cap \alpha \mathbb{Z} \neq \varnothing$.\\[0.5em]

\noindent
\textbf{5)} Montrez que si
\[
\sum_{n \ge 0} P(S_n \in I) < \infty,
\]
alors aucun point de $\alpha \mathbb{Z}$ n’est récurrent.\\[0.5em]

\noindent
Nous supposons désormais
\[
\sum_{n \ge 0} P(S_n \in I) = \infty.
\]
6) Soit $\varepsilon > 0$ plus petit que la moitié de la longueur de $I$. Montrez que
\[
\exists x \in \alpha \mathbb{Z} \cap I, \quad \sum_{k \ge 0} P(|S_k - x| < \varepsilon) = \infty. \tag{1}
\]
Nous fixons $x, \varepsilon$ vérifiant (1), et nous définissons les événements, pour $k \in \mathbb{N}$,
\[
A_k = \left\{ |S_k - x| < \varepsilon, \; \forall n \ge 1, \; |S_{k+n} - x| \ge \varepsilon \right\}.
\]

\textbf{7)} Montrez que, pour $k \in \mathbb{N}$,
\[
\{ |S_k - x| < \varepsilon, \ \forall n \geq 1 \ |S_{n+k} - S_k| \geq 2\varepsilon \} \subset A_k,
\]
puis que
\[
P(A_k) \geq P(|S_k - x| < \varepsilon) \, P(\forall n \geq 1 \ |S_n| \geq 2\varepsilon).
\tag{2}
\]

\medskip
\noindent
\textbf{8)} a) En sommant les inégalités (2), montrez que
\[
P(|S_n - x| < \varepsilon \ \text{f. s.}) 
\geq P(\forall n \geq 1 \ |S_n| \geq 2\varepsilon) 
\left( \sum_{k \geq 0} P(|S_k - x| < \varepsilon) \right).
\]
b) En déduire que $P(\forall n \geq 1 \ |S_n| \geq 2\varepsilon) = 0.$

\medskip
\noindent
\textbf{9)} Soient $\delta, \varepsilon$ tels que $0 < \delta < \varepsilon$ et $k \in \mathbb{N}$.

\medskip
a) Montrez que
\[
\{ |S_k| < \delta, \ \forall n \geq 1 \ |S_{k+n}| \geq \varepsilon \}
\subset
\{ |S_k| < \delta, \ \forall n \geq 1 \ |S_{k+n} - S_k| \geq \varepsilon - \delta \}.
\]

\medskip
b) À l’aide de la question 8)b), montrez que
\[
P(|S_k| < \delta, \ \forall n \geq 1 \ |S_{k+n}| \geq \varepsilon) = 0.
\]

\medskip
c) En déduire que
\[
P(|S_k| < \varepsilon, \ \forall n \geq 1 \ |S_{k+n}| \geq \varepsilon) = 0.
\]

\medskip
d) Montrez enfin que
\[
\forall \varepsilon > 0, \quad P(|S_n| < \varepsilon \ \text{f. s.}) = 0.
\]

\medskip
\noindent
\textbf{10)} Conclure que $0$ est récurrent et énoncez le théorème démontré.


\end{document}

