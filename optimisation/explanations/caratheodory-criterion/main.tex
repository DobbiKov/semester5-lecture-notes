\documentclass[a4paper]{article}

\usepackage[utf8]{inputenc}
\usepackage[T1]{fontenc}
\usepackage{textcomp}
\usepackage[english]{babel}
\usepackage{float}
\usepackage{amsmath, amssymb, amsthm}


% figure support
\usepackage{import}
\usepackage{xifthen}
\pdfminorversion=7
\usepackage{pdfpages}
\usepackage{transparent}
\usepackage{hyperref}
\usepackage[margin=0.8in]{geometry}

\usepackage{setspace}
\setlength{\parindent}{0in}

\newcommand{\incfig}[1]{%
    \def\svgwidth{\columnwidth}
    \import{./figures/}{#1.pdf_tex}
}

\pdfsuppresswarningpagegroup=1

\newcommand{\N}{\mathbb{N}}
\newcommand{\R}{\mathbb{R}}
\newcommand{\Z}{\mathbb{Z}}
\newcommand{\Q}{\mathbb{Q}}

\newtheorem{theorem}{Theorem}[section]
\newtheorem{definition}{Definition}[section]
\newtheorem{eg}{Example}[section]
\newtheorem{prop}{Proposition}[section]
\newtheorem{property}{Property}[section]
\newtheorem*{notation}{Notation}
\newtheorem*{remark}{Remark}

\author{Yehor Korotenko}
\title{Carathéodory Criterion}

\begin{document}
\section{Introduction}
   In measure theory, after defining the \textbf{outer measure}(denoted as $\lambda^*$) we need to
   define sets that are measurable as not all the sets are measurable, for
   instance \textit{Vitali set}. Hence,
\begin{definition}
Let $\Omega$ be a space and with $E \subset \Omega$. $E$ is  \textbf{measurable} if and only if for all subset of $\Omega$ denoted as $A$, not necessarily measurable
\[
\lambda^*(A) = \lambda^*(A \cap E) + \lambda^*(A \cap E^c)
\] 
\end{definition}
An interesting fact about $A$ is that it may be (and usually is)
non-measurable. But why does it work in the first place even for
\textit{non-measurable} $A$? Let's start with a reminder what an outer measure and its intuition is.

\section{Outer measure}%
\label{sec:Outer measure}
\begin{definition}
    An outer measure is a map denoted as $\lambda^*: \Omega \to \R^+$ that satisfies the following properties:
    \begin{enumerate}
        \item $\lambda^*(\emptyset) = 0$
        \item $A \subset B \subset \Omega \implies \lambda^*(A) < \lambda^*(B)$ 
        \item $\lambda^*(\cup A_n) \le \sum_n \lambda^*(A_n)$
    \end{enumerate}
\end{definition}
\begin{definition}\label{defn:lebesgue-outer-measure}
    The Lebesgue outer measure of a set $E \subset R^n$ is 
    \[
        \lambda^*(E) = inf \left\{ \cup_{i \in I} V_i \right\}
    \] 
    with $(V_i)_{i \in I}$ a family of boxes such that $E \subset \cup_{i_I} V_i$.
\end{definition}

Intuitively, we look for the smallest set of measurable sets (boxes) that cover
$E$.  

\begin{figure}[H]
    \centering
    \incfig{cover-example2}
    \caption{Searching for infinum of covers of $E$. \
        The black space is our
        set $E$, blue rectangles are the boxes of the  $(V_i)_{i \in I}$
        family. \
        On the right, the rectangles are smaller lead to less amount of white spaces taking into account.}
    \label{fig:cover-example2}
\end{figure}

That is the intuitive way of the \textit{Lebesgue Outer Measure} (see the
definition ~\ref{defn:lebesgue-outer-measure}). We look for smaller boxes and
better placements to cover over set $E$ in order to find the smallest size of
such cover.

Before moving on, let's see an example of non-measurable set (in our simplified sense) and measurable set (in the same sense):

\end{document}
