\documentclass[a4paper]{article}

\usepackage[utf8]{inputenc}
\usepackage[T1]{fontenc}
\usepackage{textcomp}
\usepackage[english]{babel}
\usepackage{float}
\usepackage{amsmath, amssymb, amsthm}

\usepackage[margin=0.8in]{geometry}
\usepackage{fancyhdr}
\pagestyle{fancy}
\fancyhf{} % sets both header and footer to nothing
\renewcommand{\headrulewidth}{0pt}
\fancyhead{}
\fancyfoot[R]{Yehor Korotenko}
\fancyfoot[C]{\thepage}
\fancyfoot[L]{Carathéodory criterion intuition}


% figure support
\usepackage{import}
\usepackage{xifthen}
\pdfminorversion=7
\usepackage{pdfpages}
\usepackage{transparent}
\usepackage{hyperref}

\usepackage{setspace}
\setlength{\parindent}{0in}

\newcommand{\incfig}[1]{%
    \def\svgwidth{\columnwidth}
    \import{./figures/}{#1.pdf_tex}
}

\pdfsuppresswarningpagegroup=1

\newcommand{\N}{\mathbb{N}}
\newcommand{\R}{\mathbb{R}}
\newcommand{\Z}{\mathbb{Z}}
\newcommand{\Q}{\mathbb{Q}}

\newtheorem{theorem}{Theorem}[section]
\newtheorem{definition}{Definition}[section]
\newtheorem{eg}{Example}[section]
\newtheorem{prop}{Proposition}[section]
\newtheorem{property}{Property}[section]
\newtheorem*{notation}{Notation}
\newtheorem*{remark}{Remark}

\author{Yehor Korotenko}
\title{Carathéodory Criterion}

\begin{document}
\section{Introduction}
   In measure theory, after defining the \textbf{outer measure}(denoted as $\lambda^*$) we need to
   define sets that are measurable as not all the sets are measurable, for
   instance \textit{Vitali set}. Hence,
\begin{definition}
Let $\Omega$ be a space and with $E \subset \Omega$. $E$ is  \textbf{measurable} if and only if for all subset of $\Omega$ denoted as $A$, not necessarily measurable
\[
\lambda^*(A) = \lambda^*(A \cap E) + \lambda^*(A \cap E^c)
\] 
\end{definition}
An interesting fact about $A$ is that it may be (and usually is)
non-measurable. But why does it work in the first place even for
\textit{non-measurable} $A$? Let's start with a reminder what an outer measure and its intuition is.

\section{Outer measure}%
\label{sec:Outer measure}
\begin{definition}
    An outer measure is a map denoted as $\lambda^*: \Omega \to \R^+$ that satisfies the following properties:
    \begin{enumerate}
        \item $\lambda^*(\emptyset) = 0$
        \item $A \subset B \subset \Omega \implies \lambda^*(A) < \lambda^*(B)$ 
        \item $\lambda^*(\cup A_n) \le \sum_n \lambda^*(A_n)$
    \end{enumerate}
\end{definition}
\begin{definition}\label{defn:lebesgue-outer-measure}
    The Lebesgue outer measure of a set $E \subset R^n$ is 
    \[
        \lambda^*(E) = inf \left\{ \cup_{i \in I} V_i \right\}
    \] 
    with $(V_i)_{i \in I}$ a family of boxes such that $E \subset \cup_{i_I} V_i$.
\end{definition}

Intuitively, we look for the smallest set of measurable sets (boxes) that cover
$E$.  

\begin{figure}[H]
    \centering
    \incfig{cover-example2}
    \caption{Searching for infinum of covers of $E$. \
        The black space is our
        set $E$, blue rectangles are the boxes of the  $(V_i)_{i \in I}$
        family. \
        On the right, the rectangles are smaller lead to less amount of white spaces taking into account.}
    \label{fig:cover-example2}
\end{figure}

That is the intuitive way of the \textit{Lebesgue Outer Measure} (see the
definition ~\ref{defn:lebesgue-outer-measure}). We look for smaller boxes and
better placements to cover over set $E$ in order to find the smallest size of
such cover.

In the rest of this explanation we suppose that the blue box on the right of
the Figure \ref{fig:cover-example2} is the smallest box we can take to cover any set
to calculate the outer measure.

\section{Carathéodory criterion}
Let's introduce non-measurable set (in our simplified
sense) and measurable set (in the same sense):
\begin{figure}[H]
    \centering
    \incfig{non-measurable-and-measurable}
    \caption{The set on the left is non measurable because it is not "pretty
    enough" to cover it with boxes. However, the one on the right is covered
with the boxes ideally without any white space.}
    \label{fig:non-measurable-and-measurable}
\end{figure}

\begin{remark}
    The "unpretty" set on the left from the Figure
    \ref{fig:non-measurable-and-measurable} is actually measurable by Lebesgue
    Measure. However, we introduce it as a metaphor to overly complex structure 
    for an example of "non-measurable".
\end{remark}

This unprettiness on the left of the non measurable set is the key to
understand the Carathéodory criterion, let's take a set $A$ and draw them
together with "unpretty" $E$ :

\begin{figure}[H]
    \centering
    \incfig{a-and-unpretty-e}
    \caption{The set $A$ with the set  $E$}
    \label{fig:a-and-unpretty-e}
\end{figure}

Let's take a look at how to approximate its outer measure with boxes:
\begin{figure}[H]
    \centering
    \incfig{a-outer-measure-by-boxes}
    \caption{a-outer-measure-by-boxes}
    \label{fig:a-outer-measure-by-boxes}
\end{figure}

If $E$ is measurable, then for our chosen  $A$, we must have
 \[
\lambda^*(A) = \lambda^*(A \cap E) + \lambda^*(A \cap E^c)
\] 

That is to say, that if we split $A$ on the parts of  $E$ and its complement, the sum of the measures of the splitted parts equals to the measure of $A$, let's take a look on the split of $A$:
\begin{figure}[H]
    \centering
    \incfig{split-of-a}
    \caption{The set $A$ splitted by $E$}
    \label{fig:split-of-a}
\end{figure}

Let's now approxiamte the outer measure.

\begin{figure}[H]
    \centering
    \incfig{split-of-a-outer-measure}
    \caption{Approximation of the outer measure of the two parts of the set $A$}
    \label{fig:split-of-a-outer-measure}
\end{figure}
As you can see, those "unpretty" parts of $E$ divide  $A$ into two subsets with
"branches" that require more boxes to cover them thus we get inequality:
 \[
\lambda^*(A) < \lambda^*(A \cap E) + \lambda^*(A \cap E^c)
\] 

Let's now see an example with a measurable  $E$.

\begin{figure}[H]
    \centering
    \incfig{split-of-a-by-measurable-e}
    \caption{$A$, measurable $E$ and the split of $A$ by $E$}
    \label{fig:split-of-a-by-measurable-e}
\end{figure}

Now the border of the split look nicer and with the cover:
\begin{figure}[H]
    \centering
    \incfig{cover-of-a-splitted-by-measurable-e}
    \caption{Coverage of the two parts of the $A$}
    \label{fig:cover-of-a-splitted-by-measurable-e}
\end{figure}

As you can see, the nice border allows us to use the same number of boxes as we needed to cover $A$, thus we get an equality:
 \[
\lambda^*(A) = \lambda^*(A \cap E) + \lambda^*(A \cap E^c)
\] 

\section{Important nuances}%
\label{sec:Important nuances}

\begin{itemize}
    \item 
        The "Non-Measurable" Drawings Are Just Analogies: This is the most significant
        simplification. The "unpretty" black blob shown in the figures (e.g., Figure \ref{fig:a-and-unpretty-e})
        is, technically, a Lebesgue measurable set.  Truly non-measurable sets, like
        the Vitali set, are so pathologically complex that they cannot be drawn. They
        are composed of a completely disconnected "dust" of points. The drawing is a
        metaphor for the extreme boundary complexity that causes the Carathéodory
        criterion to fail.
    \item 
        The Covering Process: The presented process of "covering" oversimplifies the
        idea of real covering and the measure, that of an outer measure by
        imagining a single grid of "the smallest box". In reality, the definition of
        outer measure involves finding the infimum—the greatest lower bound—of the sum
        of volumes of all possible countable collections of boxes that cover the set,
        with boxes of any size and position. The core idea of covering presented here
        efficiency still holds true under this precise definition.
\end{itemize}




\section{Conclusion}%
\label{sec:Conclusion}
The Carathéodory criterion is a \textbf{definition} of a \textit{measurable}
set thus is a fundamental part of the Measure Theory. This explanation may
\textit{not} be correct or rigourous. However, it should help to find an
intuition behind the concept that seem to be unintuitive.

\end{document}
