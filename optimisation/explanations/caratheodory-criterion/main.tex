\documentclass[a4paper]{article}

\usepackage[utf8]{inputenc}
\usepackage[T1]{fontenc}
\usepackage{textcomp}
\usepackage[english]{babel}
\usepackage{amsmath, amssymb, amsthm}


% figure support
\usepackage{import}
\usepackage{xifthen}
\pdfminorversion=7
\usepackage{pdfpages}
\usepackage{transparent}
\usepackage{hyperref}
\usepackage[margin=0.8in]{geometry}

\usepackage{setspace}
\setlength{\parindent}{0in}

\newcommand{\incfig}[1]{%
    \def\svgwidth{\columnwidth}
    \import{./figures/}{#1.pdf_tex}
}

\pdfsuppresswarningpagegroup=1

\newcommand{\N}{\mathbb{N}}
\newcommand{\R}{\mathbb{R}}
\newcommand{\Z}{\mathbb{Z}}
\newcommand{\Q}{\mathbb{Q}}

\newtheorem{theorem}{Theorem}[section]
\newtheorem{definition}{Definition}[section]
\newtheorem{eg}{Example}[section]
\newtheorem{prop}{Proposition}[section]
\newtheorem{property}{Property}[section]
\newtheorem*{notation}{Notation}
\newtheorem*{remark}{Remark}

\author{Yehor Korotenko}
\title{Carathéodory Criterion}

\begin{document}
\maketitle
\section{Introduction}
   In measure theory, after defining the \textbf{outer measure}(denoted as $\lambda^*$) we need to
   define sets that are measurable as not all the sets are measurable, for
   instance \textit{Vitali set}. Hence,
\begin{definition}
Let $\Omega$ be a space and with $E \subset \Omega$. $E$ is  \textbf{measurable} if and only if for all subset of $\Omega$ denoted as $A$, not necessarily measurable
\[
\lambda^*(A) = \lambda^*(A \cap E) + \lambda^*(A \cap E^c)
\] 
\end{definition}
\end{document}
