\documentclass[a4paper]{report}
\usepackage[backend=bibtex]{biblatex}
\usepackage[toc,page]{appendix}
% \addbibresource{refs.bib}
\usepackage[utf8]{inputenc}
\usepackage[T1]{fontenc}
\usepackage{textcomp}

\usepackage{url}

\usepackage{hyperref}
\hypersetup{
    colorlinks,
    linkcolor={black},
    citecolor={black},
    urlcolor={blue!80!black}
}

\usepackage{graphicx}
\usepackage{wrapfig}
\usepackage{adjustbox}
\usepackage{float}
\usepackage[usenames,dvipsnames]{xcolor}

\usepackage{listings}

\lstset{
    language=Python,
    basicstyle=\ttfamily\footnotesize,
    keywordstyle=\color{blue},
    stringstyle=\color{red},
    commentstyle=\color{gray},
    showstringspaces=false,
    frame=single,
    numbers=left,
    numberstyle=\tiny,
    breaklines=true,
    tabsize=4
}

% \usepackage{cmbright}

\usepackage{amsmath, amsfonts, mathtools, amsthm, amssymb}
\usepackage{mathrsfs}
\usepackage{cancel}

\newcommand\N{\ensuremath{\mathbb{N}}}
\newcommand\R{\ensuremath{\mathbb{R}}}
\newcommand\Z{\ensuremath{\mathbb{Z}}}
\renewcommand\O{\ensuremath{\emptyset}}
\newcommand\Q{\ensuremath{\mathbb{Q}}}
\newcommand\C{\ensuremath{\mathbb{C}}}
\let\implies\Rightarrow
\let\impliedby\Leftarrow
\let\iff\Leftrightarrow
\let\epsilon\varepsilon

% horizontal rule
\newcommand\hr{
    \noindent\rule[0.5ex]{\linewidth}{0.5pt}
}

\usepackage{tikz}
% \usepackage{tikzmark}
\usepackage{pgfplots}
\usepackage{tikz-cd}

\usetikzlibrary{calc, arrows.meta, positioning, angles, quotes, patterns}

% theorems
\usepackage{thmtools}
\usepackage{thm-restate}
\usepackage[framemethod=TikZ]{mdframed}
\mdfsetup{skipabove=1em,skipbelow=0em, innertopmargin=12pt, innerbottommargin=8pt}

\theoremstyle{definition}

\makeatletter

\declaretheoremstyle[
    headfont=\bfseries\sffamily\color{ForestGreen!70!black}, bodyfont=\normalfont,
    mdframed={
        linewidth=2pt,
        rightline=false, topline=false, bottomline=false,
        linecolor=ForestGreen, backgroundcolor=ForestGreen!5,
    }
]{thmgreenbox}

\declaretheoremstyle[
    headfont=\bfseries\sffamily\color{NavyBlue!70!black}, bodyfont=\normalfont,
    mdframed={
        linewidth=2pt,
        rightline=false, topline=false, bottomline=false,
        linecolor=NavyBlue, backgroundcolor=NavyBlue!5,
    }
]{thmbluebox}

\declaretheoremstyle[
    headfont=\bfseries\sffamily\color{NavyBlue!70!black}, bodyfont=\normalfont,
    mdframed={
        linewidth=2pt,
        rightline=false, topline=false, bottomline=false,
        linecolor=NavyBlue
    }
]{thmblueline}

\declaretheoremstyle[
    headfont=\bfseries\sffamily\color{RawSienna!70!black}, bodyfont=\normalfont,
    mdframed={
        linewidth=2pt,
        rightline=false, topline=false, bottomline=false,
        linecolor=RawSienna, backgroundcolor=RawSienna!5,
    }
]{thmredbox}

\declaretheoremstyle[
    headfont=\bfseries\sffamily\color{RawSienna!70!black}, bodyfont=\normalfont,
    numbered=no,
    mdframed={
        linewidth=2pt,
        rightline=false, topline=false, bottomline=false,
        linecolor=RawSienna, backgroundcolor=RawSienna!1,
    },
    qed=\qedsymbol
]{thmproofbox}

\declaretheoremstyle[
    headfont=\bfseries\sffamily\color{NavyBlue!70!black}, bodyfont=\normalfont,
    numbered=no,
    mdframed={
        linewidth=2pt,
        rightline=false, topline=false, bottomline=false,
        linecolor=NavyBlue, backgroundcolor=NavyBlue!1,
    },
]{thmexplanationbox}

\declaretheorem[numberwithin=chapter, style=thmgreenbox, name=Definition]{definition}
\declaretheorem[sibling=definition, style=thmredbox, name=Corollary]{corollary}
\declaretheorem[sibling=definition, style=thmredbox, name=Proposition]{prop}
\declaretheorem[sibling=definition, style=thmredbox, name=Theorem]{theorem}
\declaretheorem[sibling=definition, style=thmredbox, name=Lemma]{lemma}
\declaretheorem[sibling=definition, style=thmbluebox,  name=Example]{eg}
\declaretheorem[sibling=definition, style=thmbluebox,  name=Nonexample]{noneg}
\declaretheorem[sibling=definition, style=thmblueline, name=Remark]{remark}




\declaretheorem[numbered=no, style=thmexplanationbox, name=Proof]{explanation}
\declaretheorem[numbered=no, style=thmproofbox, name=Proof]{preuve}
\declaretheorem[style=thmbluebox,  numbered=no, name=Exercise]{ex}
\declaretheorem[style=thmblueline, numbered=no, name=Note]{note}

% \renewenvironment{proof}[1][\proofname]{\begin{replacementproof}}{\end{replacementproof}}

% \AtEndEnvironment{eg}{\null\hfill$\diamond$}%

\newtheorem*{uovt}{UOVT}
\newtheorem*{notation}{Notation}
\newtheorem*{previouslyseen}{As previously seen}
\newtheorem*{problem}{Problem}
\newtheorem*{observe}{Observe}
\newtheorem*{property}{Property}
\newtheorem*{intuition}{Intuition}


\declaretheoremstyle[
    headfont=\bfseries\sffamily\color{RawSienna!70!black}, bodyfont=\normalfont,
    mdframed={
        linewidth=2pt,
        rightline=false, topline=false, bottomline=false,
        linecolor=RawSienna, backgroundcolor=RawSienna!5,
    }
]{todo}
\declaretheorem[numbered=no, style=todo, name=TODO]{TODO}


\usepackage{etoolbox}

\AtEndEnvironment{vb}{\null\hfill$\diamond$}%
\AtEndEnvironment{intermezzo}{\null\hfill$\diamond$}%




% http://tex.stackexchange.com/questions/22119/how-can-i-change-the-spacing-before-theorems-with-amsthm
% \def\thm@space@setup{%
%   \thm@preskip=\parskip \thm@postskip=0pt
% }

\usepackage{xifthen}

\def\testdateparts#1{\dateparts#1\relax}
\def\dateparts#1 #2 #3 #4 #5\relax{
    \marginpar{\small\textsf{\mbox{#1 #2 #3 #5}}}
}

\def\@lesson{}%
\newcommand{\lesson}[3]{
    \ifthenelse{\isempty{#3}}{%
        \def\@lesson{Lecture #1}%
    }{%
        \def\@lesson{Lecture #1: #3}%
    }%
    \subsection*{\@lesson}
    \testdateparts{#2}
}

% fancy headers
\usepackage{fancyhdr}
\pagestyle{fancy}

% \fancyhead[LE,RO]{Gilles Castel}
\fancyhead[RO,LE]{\@lesson}
\fancyhead[RE,LO]{}
\fancyfoot[LE,RO]{\thepage}
\fancyfoot[C]{\leftmark}
\renewcommand{\headrulewidth}{0pt}

\makeatother

% figure support (https://castel.dev/post/lecture-notes-2)
\usepackage{import}
\usepackage{xifthen}
\pdfminorversion=7
\usepackage{pdfpages}
\usepackage{transparent}
\usepackage[margin=0.8in]{geometry}
\newcommand{\incfig}[1]{%
    \def\svgwidth{\columnwidth}
    \import{./figures/}{#1.pdf_tex}
}

% %http://tex.stackexchange.com/questions/76273/multiple-pdfs-with-page-group-included-in-a-single-page-warning
\pdfsuppresswarningpagegroup=1
\pgfplotsset{compat=1.11}
\usepackage{subcaption}

\author{Yehor Korotenko}

\newcommand{\scalar}[2]{\langle #1, #2 \rangle}
\newcommand{\scalair}[1]{\left\langle #1 \right\rangle}

% fancy chapters
\usepackage{lipsum}
\usepackage[Lenny]{fncychap}
\ChNameUpperCase
\ChNumVar{\fontsize{40}{42}\usefont{OT1}{ptm}{m}{n}\selectfont}
\ChTitleVar{\Large\sc}



\title{Droit civil}


\begin{document}
\maketitle
\begin{abstract}
   %  Ce sont mes notes prise pour le cours d'algèbre linéaire 2 à l'Univérsité Paris-Saclay. La partie essentielle de ces notes sont référencé du livre "Algèbre Linéaire" écris par Joseph Grifone \cite{grifone}.
   %
   % Mes notes d'autres matières sont accessibles sur mon site: \href{https://dobbikov.com/lecture_notes}{dobbikov.com}

  % Ces notes sont traduites en Ukrainian et Anglais en utilisant l'outil \texttt{sci-trans-git} \cite{korotenko-sci-trans-git}
    Ce sont mes notes prise pour le cours du droit civil à l'Univérsité Panthéon-Sorbonne. Je remercie à Denys Vostrikov pour son aide et pour son existence.
\end{abstract}
\tableofcontents
\chapter{Premier chapitre}\label{ch:first-chapter}
\section{Introduction}%
\label{sec:Introduction}

Nous sommes transporté en 1914. L'acte d'enfance en vue. Le registre des actes de décès.


- Jusqu'à quel âge on est enfant? - On sera toujours enfants de nos parents.

Pour s'assurer de l'exactitude des informations déclarées, l'officier de l'état civil peut demander la vérification des données à caractère personnel du défunt auprès du dépositaire de l'acte de naissance ou, à défaut d'acte de naissance détenu en France, de l'acte de mariage.


Dans les d'états (notamment des décès) civil on dit la place et la date = l'affilation de différents rôle que nous allons prendre en cours de notre \ldots

Tous les humains du livre un sin opté de la personnalité juridique et on peut voir le lien de droit qui dicte par rapport des autres, par exemple familial.

titre 1er, ch. 3 - de l'examen des Cour d'appel caractéristiques génétiques d'une personne et de l'identification d'une personne ses empreintes génétiques.

titre 9 --- de la majorité des majeurs protégés par les loi ---
ch 2 - -l'autorité parentale relativement aux biens de l'enfant 

Les biens appartient à l'enfant est les droit sont protégés par le \textbf{Code Civil}

\begin{remark}
    Article 16 - à apprendre par cœur - concernant de la primauté de la personne
    \begin{itemize}
        \item Il s'applique strictement à l'être humain né,
        \item les deux sens sont cumulées.
        \item pourquoi l'être humain est douté par la distinction de la juridique.
    \end{itemize}
\end{remark}

Article 1842: Les sociétés autres que les sociétés en participation visées au
chapitre $3$ et que les sociétés de libre partenariat spéciales mentionnées à
\underline{l'article L. 214-162-13 du code monétaire et financier} jouissent de
la personnalité morale à compter de leur immatriculation.

Jusqu'à l'immatriculation, les rapports entre les associés sont régis par le
contrat de société et par les principes généraux du droit applicable aux
contrats et obligations.

La concept de personnalité morale n'est pas suffisamment étudié et bien développé

L'enfant conçu et non encore né n'est pas doté de personnalité juridique.

être humains et personnes ne sont pas liés
on s'interroger sur la personnalité en notion tant que la personne
dans quel moment de l'être humain est doté par la notion de personnalité juridique


après la révélation des expériences nazies incontrôlés sur des êtres humains dans le cadre du procès de Nurnberg - la définition - BIOÉTHIQUE, dans les années 60.

\section{La personne et l'être humain}%
\label{sec:La personne et l'être humain}

\begin{enumerate}
    \item Se redouble en deux question fond
        \begin{enumerate}
            \item si tous les personnes sont l'être humain?
            \item toutes les personnes sont obligatoirement l'être humains?
        \end{enumerate}
        Interrogation concernant l'égalité des personnes juridiques

        Déclaration des Droits de l'Homme et du Citoyen de 1789 chaque eh ont la
        personnalité juridique par le fait de la naissance;
    \item 1848 - abolition d'esclavage \\
        l'exploitation économiques d'être humains est indirectement mis en
        pause par les clauses économiques évoquées par l'uns par rapport aux
        autres. \\
        La vérification économiques du corps.
\end{enumerate}


% \nocite{*}
% \printbibliography

\end{document}
