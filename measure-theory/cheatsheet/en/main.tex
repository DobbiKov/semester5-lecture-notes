\documentclass[10pt, a4paper, landscape]{article}
\usepackage[utf8]{inputenc}
\usepackage[T1]{fontenc}
\usepackage{geometry}
\usepackage{amsmath, amssymb, amsthm}
\usepackage{multicol}
\usepackage{enumitem}
\usepackage{xcolor}
\usepackage{titlesec}
\usepackage{fancyhdr}
\usepackage{tcolorbox}

% Page Layout
\geometry{top=1cm, bottom=1cm, left=1cm, right=1cm}
\setlength{\parindent}{0pt}
\setlength{\parskip}{2pt}
\pagestyle{fancy}
\fancyhf{}
\renewcommand{\headrulewidth}{0pt}
\rfoot{\footnotesize Generated from Chapters 3 \& 4: Intégrale de Lebesgue}

% Colors
\definecolor{secblue}{RGB}{0, 50, 120}
\definecolor{boxbg}{RGB}{245, 247, 250}
\definecolor{boxframe}{RGB}{0, 50, 120}

% Section Styling
\titleformat{\section}
  {\normalfont\large\bfseries\color{secblue}}{}{0em}{}[\hrule]
\titlespacing*{\section}{0pt}{5pt}{2pt}

% Custom Box Environment
\newtcolorbox{cheatbox}[1]{
  colback=boxbg,
  colframe=boxframe,
  coltitle=white,
  title=\textbf{#1},
  boxrule=0.5pt,
  arc=2pt,
  left=2pt, right=2pt, top=2pt, bottom=2pt,
  fonttitle=\small\sffamily
}

% Math shortcuts
\newcommand{\R}{\mathbb{R}}
\newcommand{\N}{\mathbb{N}}
\newcommand{\Lspace}{\mathcal{L}}
\newcommand{\Eplus}{\mathcal{E}^+}
\newcommand{\Ind}{\mathbf{1}}
\newcommand{\dlam}{d\lambda}

\begin{document}

\begin{center}
    {\Huge \textbf{\color{secblue} Lebesgue Integration Cheatsheet}} \\
    \small Based on \textit{Intégrale de Lebesgue sur $\R$} and \textit{sur $\R^d$}
\end{center}

\begin{multicols*}{3}

% ==========================================
% COLUMN 1: Integration on R
% ==========================================

\section{Integration on $\R$}

\begin{cheatbox}{1. Simple Functions (Fonctions Étagées)}
A function $f: X \to [0, +\infty]$ is \textbf{simple positive} ($\Eplus(X)$) if it takes a finite number of values $\{c_1, \dots, c_n\}$ on measurable sets $A_j$:
\[ f = \sum_{j=1}^n c_j \Ind_{A_j} \]
\textbf{Integral Definition:}
\[ \int_X f \, \dlam = \sum_{c \in f(X)} c \lambda(f^{-1}(\{c\})) \]
With convention $0 \times \infty = 0$.
\end{cheatbox}

\begin{cheatbox}{2. Measurable Positive Functions}
For a measurable $f: X \to [0, +\infty]$:
\[ \int_X f \, \dlam = \sup \left\{ \int_X \varphi \, \dlam : \varphi \in \Eplus(X), \varphi \le f \right\} \]
\textbf{Approximation:} Every measurable $f \ge 0$ is the limit of an increasing sequence of simple functions $\varphi_n \nearrow f$.
\end{cheatbox}

\begin{cheatbox}{3. Integrable Functions ($\Lspace^1$)}
A measurable function $f: X \to \R$ is \textbf{integrable} (denoted $f \in \Lspace^1(X)$) if:
\[ \int_X |f| \, \dlam < +\infty \]
\textbf{Definition of Integral:}
\[ \int_X f \, \dlam = \int_X f^+ \, \dlam - \int_X f^- \, \dlam \]
where $f^+ = \max(f, 0)$ and $f^- = \max(-f, 0)$.
\end{cheatbox}

\begin{cheatbox}{Key Properties}
\begin{itemize}[leftmargin=*]
    \item \textbf{Linearity:} $\int (\alpha f + \beta g) = \alpha \int f + \beta \int g$.
    \item \textbf{Monotonicity:} $f \le g \implies \int f \le \int g$.
    \item \textbf{Triangle Inequality:} $|\int f| \le \int |f|$.
    \item \textbf{Null Sets:} $\int_N f = 0$ if $\lambda(N)=0$.
    \item \textbf{Vanishing Integral:} For $f \ge 0$, $\int f = 0 \iff f = 0$ a.e.
\end{itemize}
\end{cheatbox}

\begin{cheatbox}{Markov Inequality}
For measurable $f \ge 0$ and $\alpha > 0$:
\[ \lambda(\{x \in X : f(x) \ge \alpha\}) \le \frac{1}{\alpha} \int_X f \, \dlam \]
\end{cheatbox}

\begin{cheatbox}{Relation to Riemann}
If $f: [a,b] \to \R$ is continuous (or Riemann-integrable), it is Lebesgue integrable and the integrals coincide.
\end{cheatbox}

% ==========================================
% COLUMN 2: Limit Theorems & Parameters
% ==========================================

\section{Limit Theorems}

\begin{cheatbox}{Monotone Convergence (Beppo-Levi)}
If $(f_n)$ is an \textbf{increasing} sequence of measurable \textbf{positive} functions ($0 \le f_n \le f_{n+1}$):
\[ \lim_{n \to \infty} \int_X f_n \, \dlam = \int_X \left( \lim_{n \to \infty} f_n \right) \, \dlam \]
\textit{Note: Allows limit/integral exchange even if integral is infinite.}
\end{cheatbox}

\begin{cheatbox}{Fatou's Lemma}
For any sequence of measurable \textbf{positive} functions $(f_n \ge 0)$:
\[ \int_X \liminf_{n \to \infty} f_n \, \dlam \le \liminf_{n \to \infty} \int_X f_n \, \dlam \]
\end{cheatbox}

\begin{cheatbox}{Dominated Convergence Theorem (DCT)}
Let $(f_n)$ be a sequence of measurable functions such that:
\begin{enumerate}[leftmargin=*, nosep]
    \item $f_n(x) \to f(x)$ for almost every $x$.
    \item \textbf{Domination:} There exists $g \in \Lspace^1(X)$ such that $|f_n(x)| \le g(x)$ for all $n$, a.e. $x$.
\end{enumerate}
Then $f \in \Lspace^1(X)$ and:
\[ \lim_{n \to \infty} \int_X f_n \, \dlam = \int_X f \, \dlam \]
\end{cheatbox}

\section{Integrals with Parameters}
Let $F(u) = \int_X f(u,x) \, \dlam(x)$ for $u \in I$.

\begin{cheatbox}{Continuity}
$F$ is continuous on $I$ if:
\begin{itemize}[leftmargin=*, nosep]
    \item $x \mapsto f(u,x)$ is measurable.
    \item $u \mapsto f(u,x)$ is continuous a.e.
    \item \textbf{Domination:} $|f(u,x)| \le g(x)$ with $g \in \Lspace^1$.
\end{itemize}
\end{cheatbox}

\begin{cheatbox}{Differentiability (Leibniz Rule)}
$F$ is differentiable and $F'(u) = \int \frac{\partial f}{\partial u} \dlam$ if:
\begin{itemize}[leftmargin=*, nosep]
    \item $x \mapsto f(u,x)$ is integrable.
    \item $u \mapsto f(u,x)$ is differentiable a.e.
    \item \textbf{Domination:} $|\frac{\partial f}{\partial u}(u,x)| \le g(x)$ with $g \in \Lspace^1$.
\end{itemize}
\end{cheatbox}

% ==========================================
% COLUMN 3: R^d, Fubini, Change of Vars
% ==========================================

\section{Integration on $\R^d$}

\begin{cheatbox}{Measure on $\R^d$}
Built from products of intervals (pavés).
\begin{itemize}[leftmargin=*, nosep]
    \item $\lambda_d(A \times B) = \lambda_p(A) \lambda_{d-p}(B)$.
    \item Sets of dimension $k < d$ (e.g., lines in $\R^2$) have measure 0.
    \item Invariant by translation and rotation (isometries).
\end{itemize}
\end{cheatbox}

\begin{cheatbox}{Tonelli's Theorem (Positive Functions)}
If $f: \R^{p+q} \to [0, +\infty]$ is measurable \textbf{positive}:
\begin{align*}
\int_{\R^{p+q}} f \, \dlam 
&= \int_{\R^q} \left( \int_{\R^p} f(x,y) dx \right) dy \\
&= \int_{\R^p} \left( \int_{\R^q} f(x,y) dy \right) dx
\end{align*}
\textit{Use this to check integrability (i.e., if integral is finite).}
\end{cheatbox}

\begin{cheatbox}{Fubini's Theorem (Integrable Functions)}
If $f: \R^{p+q} \to \R$ is \textbf{integrable} ($f \in \Lspace^1$), then:
\begin{itemize}[leftmargin=*, nosep]
    \item The slices $x \mapsto f(x,y)$ are in $\Lspace^1$ for a.e. $y$.
    \item The integral order can be swapped (same formula as Tonelli).
\end{itemize}
\textbf{Standard Strategy:} Use Tonelli on $|f|$ to prove $f \in \Lspace^1$, then Fubini to compute.
\end{cheatbox}

\section{Change of Variables}

\begin{cheatbox}{Linear Transformation}
If $T: \R^d \to \R^d$ is an isomorphism and $f \in \Lspace^1$:
\[ \int_{\R^d} f(y) \, dy = \int_{\R^d} f(Tx) \, |\det T| \, dx \]
\end{cheatbox}

\begin{cheatbox}{General Diffeomorphism ($C^1$)}
Let $\Phi: U \to V$ be a $C^1$-diffeomorphism between open sets in $\R^d$. 
$f \in \Lspace^1(V)$ iff $(f \circ \Phi) |\det J_\Phi| \in \Lspace^1(U)$.
\[ \int_V f(y) \, dy = \int_U f(\Phi(x)) \, |\det J_\Phi(x)| \, dx \]
where $J_\Phi(x)$ is the Jacobian matrix.
\end{cheatbox}

\begin{cheatbox}{Example: Polar Coordinates ($\R^2$)}
$\Phi(r, \theta) = (r \cos \theta, r \sin \theta)$. 
Determinant: $|\det J_\Phi| = r$.
\[ \int_{\R^2} f \, dx dy = \int_0^\infty \int_{-\pi}^\pi f(r\cos\theta, r\sin\theta) \, r \, d\theta dr \]
\textit{Application:} $\int_\R e^{-x^2} dx = \sqrt{\pi}$.
\end{cheatbox}

\end{multicols*}
\end{document}
