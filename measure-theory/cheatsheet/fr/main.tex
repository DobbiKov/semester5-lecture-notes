% --- CHUNK_METADATA_START ---
% needs_review: True
% src_checksum: 15bf3739862ffe3fc76981b7c2578512cc30d87e3c32bdb96c84fd753a0e6628
% --- CHUNK_METADATA_END ---
\documentclass[10pt, a4paper, landscape]{article}
\usepackage[utf8]{inputenc}
\usepackage[T1]{fontenc}
\usepackage{geometry}
\usepackage{amsmath, amssymb, amsthm}
\usepackage{multicol}
\usepackage{enumitem}
\usepackage{xcolor}
\usepackage{titlesec}
\usepackage{fancyhdr}
\usepackage{tcolorbox}

%  Mise en page
\geometry{top=1cm, bottom=1cm, left=1cm, right=1cm}
\setlength{\parindent}{0pt}
\setlength{\parskip}{2pt}
\pagestyle{fancy}
\fancyhf{}
\renewcommand{\headrulewidth}{0pt}
\rfoot{\footnotesize Généré à partir des Chapitres 3 \& 4 : Intégrale de Lebesgue}

%  Couleurs
\definecolor{secblue}{RGB}{0, 50, 120}
\definecolor{boxbg}{RGB}{245, 247, 250}
\definecolor{boxframe}{RGB}{0, 50, 120}

%  Style de section
\titleformat{\section}
  {\normalfont\large\bfseries\color{secblue}}{}{0em}{}[\hrule]
\titlespacing*{\section}{0pt}{5pt}{2pt}

%  Environnement de boîte personnalisé
\newtcolorbox{cheatbox}[1]{
  colback=boxbg,
  colframe=boxframe,
  coltitle=white,
  title=\textbf{#1},
  boxrule=0.5pt,
  arc=2pt,
  left=2pt, right=2pt, top=2pt, bottom=2pt,
  fonttitle=\small\sffamily
}

%  Raccourcis mathématiques
\newcommand{\R}{\mathbb{R}}
\newcommand{\N}{\mathbb{N}}
\newcommand{\Lspace}{\mathcal{L}}
\newcommand{\Eplus}{\mathcal{E}^+}
\newcommand{\Ind}{\mathbf{1}}
\newcommand{\dlam}{d\lambda}% --- CHUNK_METADATA_START ---
% needs_review: True
% src_checksum: 49d34b0366eb0a68eb35c75d6d8aeb1d60be3c49779501a7e1cdf4941acdab46
% --- CHUNK_METADATA_END ---
\begin{document}% --- CHUNK_METADATA_START ---
% needs_review: True
% src_checksum: 2d4e35184ba4e0a4203ea4d6c16c5903914cda0680276ebab8b7c095735c4730
% --- CHUNK_METADATA_END ---
\begin{center}
    {\Huge \textbf{\color{secblue} Lebesgue Integration Cheatsheet}}
    {\Huge \textbf{\color{secblue} Lebesgue Integration Cheatsheet}} \\
    \small Basé sur \textit{Intégrale de Lebesgue sur $\R$} et \textit{sur $\R^d$}
\end{center}% --- CHUNK_METADATA_START ---
% needs_review: True
% src_checksum: 25becdacf338bab3d28e742d6b035b2032bf49d054d56dda1096f849c78ab0d6
% --- CHUNK_METADATA_END ---
\begin{multicols*}{3}{3}

%  ==========================================
%  COLONNE 1 : Intégration sur R
%  ==========================================


\section{Intégration sur $\R$}

\begin{cheatbox}{1. Fonctions Simples (Fonctions Étagées)}{1. Fonctions Simples (Fonctions Étagées)}
Une fonction $f: X \to [0, +\infty]$ est \textbf{positive simple} ($\Eplus(X)$) si elle prend un nombre fini de valeurs $\{c_1, \dots, c_n\}$ sur des ensembles mesurables $A_j$ :
\[ f = \sum_{j=1}^n c_j \Ind_{A_j} \]
\textbf{Définition de l'Intégrale :}
\[ \int_X f \, \dlam = \sum_{c \in f(X)} c \lambda(f^{-1}(\{c\})) \]
Avec la convention $0 \times \infty = 0$.
\end{cheatbox}

\begin{cheatbox}{2. Fonctions Positives Mesurables}{2. Fonctions Positives Mesurables}
Pour une fonction mesurable $f: X \to [0, +\infty]$ :
\[ \int_X f \, \dlam = \sup \left\{ \int_X \varphi \, \dlam : \varphi \in \Eplus(X), \varphi \le f \right\} \]
\textbf{Approximation :} Toute fonction mesurable $f \ge 0$ est la limite d'une suite croissante de fonctions simples $\varphi_n \nearrow f$.
\end{cheatbox}

\begin{cheatbox}{3. Fonctions Intégrables ($\Lspace^1$)}{3. Fonctions Intégrables ($\Lspace^1$)}
Une fonction mesurable $f: X \to \R$ est \textbf{intégrable} (notée $f \in \Lspace^1(X)$) si :
\[ \int_X |f| \, \dlam < +\infty \]
\textbf{Définition de l'Intégrale :}
\[ \int_X f \, \dlam = \int_X f^+ \, \dlam - \int_X f^- \, \dlam \]
où $f^+ = \max(f, 0)$ et $f^- = \max(-f, 0)$.
\end{cheatbox}

\begin{cheatbox}{Propriétés Clés}{Propriétés Clés}
\begin{itemize}[leftmargin=*]
    \item \textbf{Linéarité :} $\int (\alpha f + \beta g) = \alpha \int f + \beta \int g$.
    \item \textbf{Monotonie :} $f \le g \implies \int f \le \int g$.
    \item \textbf{Inégalité Triangulaire :} $|\int f| \le \int |f|$.
    \item \textbf{Ensembles Nuls :} $\int_N f = 0$ si $\lambda(N)=0$.
    \item \textbf{Intégrale Nulle :} Pour $f \ge 0$, $\int f = 0 \iff f = 0$ p.p.
\end{itemize}
\end{cheatbox}

\begin{cheatbox}{Inégalité de Markov}{Inégalité de Markov}
Pour $f \ge 0$ mesurable et $\alpha > 0$ :
\[ \lambda(\{x \in X : f(x) \ge \alpha\}) \le \frac{1}{\alpha} \int_X f \, \dlam \]
\end{cheatbox}

\begin{cheatbox}{Relation avec Riemann}{Relation avec Riemann}
Si $f: [a,b] \to \R$ est continue (ou Riemann-intégrable), elle est Lebesgue-intégrable et les intégrales coïncident.
\end{cheatbox}

%  ==========================================
%  COLONNE 2 : Théorèmes de Limite & Paramètres
%  ==========================================


\section{Théorèmes de Limite}

\begin{cheatbox}{Convergence Monotone (Beppo-Levi)}{Convergence Monotone (Beppo-Levi)}
Si $(f_n)$ est une suite \textbf{croissante} de fonctions \textbf{positives} mesurables ($0 \le f_n \le f_{n+1}$) :
\[ \lim_{n \to \infty} \int_X f_n \, \dlam = \int_X \left( \lim_{n \to \infty} f_n \right) \, \dlam \]
\textit{Remarque : Permet l'échange limite/intégrale même si l'intégrale est infinie.}
\end{cheatbox}

\begin{cheatbox}{Lemme de Fatou}{Lemme de Fatou}
Pour toute suite de fonctions \textbf{positives} mesurables $(f_n \ge 0)$ :
\[ \int_X \liminf_{n \to \infty} f_n \, \dlam \le \liminf_{n \to \infty} \int_X f_n \, \dlam \]
\end{cheatbox}

\begin{cheatbox}{Théorème de Convergence Dominée (TCD)}{Théorème de Convergence Dominée (TCD)}
Soit $(f_n)$ une suite de fonctions mesurables telle que :
\begin{enumerate}[leftmargin=*, nosep]
    \item $f_n(x) \to f(x)$ pour presque tout $x$.
    \item \textbf{Domination :} Il existe $g \in \Lspace^1(X)$ telle que $|f_n(x)| \le g(x)$ pour tout $n$, p.p. $x$.
\end{enumerate}
Alors $f \in \Lspace^1(X)$ et :
\[ \lim_{n \to \infty} \int_X f_n \, \dlam = \int_X f \, \dlam \]
\end{cheatbox}

\section{Intégrales à Paramètres}
Soit $F(u) = \int_X f(u,x) \, \dlam(x)$ pour $u \in I$.

\begin{cheatbox}{Continuité}{Continuité}
$F$ est continue sur $I$ si :
\begin{itemize}[leftmargin=*, nosep]
    \item $x \mapsto f(u,x)$ est mesurable.
    \item $u \mapsto f(u,x)$ est continue p.p.
    \item \textbf{Domination :} $|f(u,x)| \le g(x)$ avec $g \in \Lspace^1$.
\end{itemize}
\end{cheatbox}

\begin{cheatbox}{Différentiabilité (Règle de Leibniz)}{Différentiabilité (Règle de Leibniz)}
$F$ est différentiable et $F'(u) = \int \frac{\partial f}{\partial u} \dlam$ si :
\begin{itemize}[leftmargin=*, nosep]
    \item $x \mapsto f(u,x)$ est intégrable.
    \item $u \mapsto f(u,x)$ est différentiable p.p.
    \item \textbf{Domination :} $|\frac{\partial f}{\partial u}(u,x)| \le g(x)$ avec $g \in \Lspace^1$.
\end{itemize}
\end{cheatbox}

%  ==========================================
%  COLONNE 3 : R^d, Fubini, Changement de Vars
%  ==========================================


\section{Intégration sur $\R^d$}

\begin{cheatbox}{Mesure sur $\R^d$}{Mesure sur $\R^d$}
Construite à partir de produits d'intervalles (pavés).
\begin{itemize}[leftmargin=*, nosep]
    \item $\lambda_d(A \times B) = \lambda_p(A) \lambda_{d-p}(B)$.
    \item Les ensembles de dimension $k < d$ (par exemple, les lignes dans $\R^2$) ont une mesure 0.
    \item Invariante par translation et rotation (isométries).
\end{itemize}
\end{cheatbox}

\begin{cheatbox}{Théorème de Tonelli (Fonctions Positives)}{Théorème de Tonelli (Fonctions Positives)}
Si $f: \R^{p+q} \to [0, +\infty]$ est mesurable \textbf{positive} :
\begin{align*}
\int_{\R^{p+q}} f \, \dlam 
&= \int_{\R^q} \left( \int_{\R^p} f(x,y) dx \right) dy \\
&= \int_{\R^p} \left( \int_{\R^q} f(x,y) dy \right) dx
\end{align*}
\textit{Utilisez ceci pour vérifier l'intégrabilité (c'est-à-dire, si l'intégrale est finie).}
\end{cheatbox}

\begin{cheatbox}{Théorème de Fubini (Fonctions Intégrables)}{Théorème de Fubini (Fonctions Intégrables)}
Si $f: \R^{p+q} \to \R$ est \textbf{intégrable} ($f \in \Lspace^1$), alors :
\begin{itemize}[leftmargin=*, nosep]
    \item Les tranches $x \mapsto f(x,y)$ sont dans $\Lspace^1$ pour presque tout $y$.
    \item L'ordre d'intégration peut être inversé (même formule que Tonelli).
\end{itemize}
\textbf{Stratégie Standard :} Utilisez Tonelli sur $|f|$ pour prouver que $f \in \Lspace^1$, puis Fubini pour calculer.
\end{cheatbox}

\section{Changement de Variables}

\begin{cheatbox}{Transformation Linéaire}{Transformation Linéaire}
Si $T: \R^d \to \R^d$ est un isomorphisme et $f \in \Lspace^1$ :
\[ \int_{\R^d} f(y) \, dy = \int_{\R^d} f(Tx) \, |\det T| \, dx \]
\end{cheatbox}

\begin{cheatbox}{Difféomorphisme Général ($C^1$)}{Difféomorphisme Général ($C^1$)}
Soit $\Phi: U \to V$ un $C^1$-difféomorphisme entre des ensembles ouverts dans $\R^d$. 
$f \in \Lspace^1(V)$ ssi $(f \circ \Phi) |\det J_\Phi| \in \Lspace^1(U)$.
\[ \int_V f(y) \, dy = \int_U f(\Phi(x)) \, |\det J_\Phi(x)| \, dx \]
where $J_\Phi(x)$ est la matrice Jacobienne.
\end{cheatbox}


\begin{cheatbox}{Exemple : Coordonnées Polaires ($\R^2$)}
$\Phi(r, \theta) = (r \cos \theta, r \sin \theta)$. 
Déterminant : $|\det J_\Phi| = r$.
\[ \int_{\R^2} f \, dx dy = \int_0^\infty \int_{-\pi}^\pi f(r\cos\theta, r\sin\theta) \, r \, d\theta dr \]
\textit{Application :} $\int_\R e^{-x^2} dx = \sqrt{\pi}$.
\end{cheatbox}

% --- CHUNK_METADATA_START ---
% needs_review: True
% src_checksum: 2dc670e7ffe1f12aa0326631b39a0b6d72da153425c7f5f9aed627a71c1487d6
% --- CHUNK_METADATA_END ---
\end{multicols*}
\end{document}
